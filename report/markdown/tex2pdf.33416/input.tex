\documentclass[12pt,,]{report}
\usepackage{lmodern}
\usepackage{amssymb,amsmath}
\usepackage{ifxetex,ifluatex}
\usepackage{fixltx2e} % provides \textsubscript
\ifnum 0\ifxetex 1\fi\ifluatex 1\fi=0 % if pdftex
  \usepackage[T1]{fontenc}
  \usepackage[utf8]{inputenc}
\else % if luatex or xelatex
  \ifxetex
    \usepackage{mathspec}
  \else
    \usepackage{fontspec}
  \fi
  \defaultfontfeatures{Ligatures=TeX,Scale=MatchLowercase}

    \usepackage{xeCJK}
    % 中文自動換行
    \XeTeXlinebreaklocale "zh"
    % 文字的彈性間距
    \XeTeXlinebreakskip = 0pt plus 1pt
    \newfontlanguage{Chinese}{CHN}
    % 章次20級,節次16級,小節次以下14級,本文12級字
    \def\LARGE{\fontsize{20}{30}\selectfont}%章次
    \def\Large{\fontsize{16}{24}\selectfont}%節次
    \def\large{\fontsize{14}{21}\selectfont}%小節次
    \usepackage{indentfirst}
    \usepackage{CJKnumb}
    \renewcommand{\figurename}{圖}
    \renewcommand{\thefigure}{{\arabic{chapter}}.\arabic{figure}}
    \renewcommand{\tablename}{表}
    \renewcommand{\thetable}{{\arabic{chapter}}.\arabic{table}}
    %重製章節
    \renewcommand{\chaptername}{}
    \renewcommand{\thechapter}{第\CJKnumber{\arabic{chapter}}章}
    \renewcommand{\thesection}{{\arabic{chapter}}.\arabic{section}}
    \renewcommand{\thesubsection}{{\arabic{chapter}}.{\arabic{section}}.\arabic{subsection}}
    %設定行距與中英文字型
    \linespread{1}\selectfont
    \setCJKmainfont{SimSun}
    \setmainfont{Times New Roman}
    \setromanfont{Times New Roman}
    \setmonofont{Times New Roman}
    %重製章節標籤
    \usepackage{titlesec}
    \titleformat{\chapter}[block]{\LARGE\centering}{\thechapter}{0.5em}{}
    \titleformat{\section}[block]{\Large}{\thesection}{0.5em}{}
    \titleformat{\subsection}[block]{\large}{\thesubsection}{0.5em}{}
    % 重製目錄
    \usepackage{titletoc}
    \titlespacing{\chapter}{0pt}{*0}{*2}
    \titlespacing{\section}{0pt}{*1}{*1}
    \titlespacing{\subsection}{0pt}{*1}{*1}
    \titlespacing{\subsubsection}{0pt}{*1}{*1}
    \titlecontents{chapter}[0em]{}{\contentspush{\thecontentslabel}\hspace*{1em}}{}{\titlerule*[0.7pc]{.}\contentspage}
\fi
% use upquote if available, for straight quotes in verbatim environments
\IfFileExists{upquote.sty}{\usepackage{upquote}}{}
% use microtype if available
\IfFileExists{microtype.sty}{
\usepackage{microtype}
\UseMicrotypeSet[protrusion]{basicmath} % disable protrusion for tt fonts
}{}
\usepackage[margin=1in]{geometry}
\usepackage[unicode=true]{hyperref}
\hypersetup{
            pdfauthor={設計一乙 40623201 王君庭; 設計一乙 40623202 吳姍蓉; 設計一乙 40623203 蔡宜芳; 設計一乙 40623210 翁嘉宏; 設計一乙 40623211 王得榮; 設計一乙 40623212 魏有泉},
            pdfborder={0 0 0},
            breaklinks=true}
\urlstyle{same}  % don't use monospace font for urls
\ifnum 0\ifxetex 1\fi\ifluatex 1\fi=0 % if pdftex
  \usepackage[shorthands=off,main=]{babel}
\else
  \usepackage{polyglossia}
  \setmainlanguage[]{}
\fi
\usepackage{longtable,booktabs}
% Fix footnotes in tables (requires footnote package)
\IfFileExists{footnote.sty}{\usepackage{footnote}\makesavenoteenv{long table}}{}
\usepackage{graphicx,grffile}
\makeatletter
\def\maxwidth{\ifdim\Gin@nat@width>\linewidth\linewidth\else\Gin@nat@width\fi}
\def\maxheight{\ifdim\Gin@nat@height>\textheight\textheight\else\Gin@nat@height\fi}
\makeatother
% Scale images if necessary, so that they will not overflow the page
% margins by default, and it is still possible to overwrite the defaults
% using explicit options in \includegraphics[width, height, ...]{}
\setkeys{Gin}{width=\maxwidth,height=\maxheight,keepaspectratio}
\IfFileExists{parskip.sty}{%
\usepackage{parskip}
}{% else
\setlength{\parindent}{0pt}
\setlength{\parskip}{6pt plus 2pt minus 1pt}
}
\setlength{\emergencystretch}{3em}  % prevent overfull lines
\providecommand{\tightlist}{%
  \setlength{\itemsep}{0pt}\setlength{\parskip}{0pt}}
\setcounter{secnumdepth}{5}
% Redefines (sub)paragraphs to behave more like sections
\ifx\paragraph\undefined\else
\let\oldparagraph\paragraph
\renewcommand{\paragraph}[1]{\oldparagraph{#1}\mbox{}}
\fi
\ifx\subparagraph\undefined\else
\let\oldsubparagraph\subparagraph
\renewcommand{\subparagraph}[1]{\oldsubparagraph{#1}\mbox{}}
\fi

% set default figure placement to htbp
\makeatletter
\def\fps@figure{htbp}
\makeatother


\begin{document}
%Cover Start
\begin{titlepage}
\vspace{1cm}
\begin{center}
\fontsize{36}{54}\selectfont{
    國立虎尾科技大學\par
}
\fontsize{28}{42}\selectfont{機械設計工程系\par}
\fontsize{24}{36}\selectfont{計算機程式 bg1 期末報告\par}
\vspace{1.5cm}
\fontsize{20}{30}\selectfont{
    PyQt5 事件導向計算器\par
    PyQt5 Event-Driven Calculator Project\par
}
\vspace{\fill}
\fontsize{18}{27}\selectfont{
    學生:\par
    設計一乙 40623201 王君庭 \par 設計一乙 40623202 吳姍蓉 \par 設計一乙 40623203 蔡宜芳 \par 設計一乙 40623210 翁嘉宏 \par 設計一乙 40623211 王得榮 \par 設計一乙 40623212 魏有泉 \par
    指導教授:嚴家銘\par
}
\vspace{1.5cm}
\fontsize{16}{24}\selectfont{2017.12.18\par}
\end{center}
\vspace{1cm}
\end{titlepage}

\newcommand\frontmatter{
    \cleardoublepage
    \pagenumbering{roman}
}

\newcommand\mainmatter{
    \cleardoublepage
    \pagenumbering{arabic}
}

\newcommand\backmatter{
    \if@openright
        \cleardoublepage
    \else
        \clearpage
    \fi
}

%Document start

% Set page number to arabic i ii...
\frontmatter
\renewcommand{\abstractname}{\LARGE \center 摘要}
\chapter*{摘要}
\addcontentsline{toc}{chapter}{摘要}
\fontsize{14}{21}\selectfont{這裡是摘要內容。A pipe character, followed by an indented block of text
is treated as a literal block, in which newlines are preserved
throughout the block, including the final newline.

\begin{itemize}
\tightlist
\item
  以 YAML 的方式插入。
\item
  The `+' indicator says to keep newlines at the end of text blocks.
\item
  使用 Markdown 語法。
\item
  前面使用加號
\end{itemize}

本研究的重點在於 \ldots{}}


\begingroup
    \renewcommand{\contentsname}{\center 目錄 \addcontentsline{toc}{chapter}{目錄}}
    \renewcommand{\numberline}[1]{~#1\hspace*{1em}}
        \setcounter{tocdepth}{2}
    \tableofcontents
    \newcommand{\lotlabel}{表}
    \renewcommand{\listtablename}{\center 表目錄 \addcontentsline{toc}{chapter}{表目錄}}
    \renewcommand{\numberline}[1]{\lotlabel~#1\hspace*{1em}}
    \listoftables
    \newcommand{\loflabel}{圖}
    \renewcommand{\listfigurename}{\center 圖目錄 \addcontentsline{toc}{chapter}{圖目錄}}
    \renewcommand{\numberline}[1]{\loflabel~#1\hspace*{1em}}
    \listoffigures
\endgroup

% Start normal page number, 1 2 3
\mainmatter
\hypertarget{ux524dux8a00}{%
\chapter{前言}\label{ux524dux8a00}}

計算器程式期末報告前言

前言: 電腦輔助設計(Computer Aided
Design),係指運用電腦功能及特性協助使用者完成設計。
輔助係指輔佐、非主要的,意旨電腦只是相持物,而非設計主要核心,操作電腦的使用者才是主體。然而,電腦輔助設計中,並非要使用特定軟硬體才能完成工作,面對各種實際情況的考驗,在堪用工具,甚至自行打造工具完成設計。
設計是一種表達運用:口語、文字、2D、3D、數學、實體等表達方法交互運用下所完成的可交付內容,現今所謂機械設計中的互動元件泛指:固體、流體與軟體元件。
本學期計算機程式是由 python3 與 PyQt5
建立簡易的計算機,其中使用基本python3 程式語法 python3
物件導向和視窗事件驅動程式的基本概念。使用可攜隨身系統中的基本視窗命令提示指令元
(Command Prompt Commands) 編輯近端檔案及提交至遠端倉儲,第十一週分組使用
github 協同倉儲製作簡易計算機。

\hypertarget{ux53efux651cux7a0bux5f0fux7cfbux7d71ux4ecbux7d39}{%
\chapter{可攜程式系統介紹}\label{ux53efux651cux7a0bux5f0fux7cfbux7d71ux4ecbux7d39}}

可攜程式系統介紹

\hypertarget{ux555fux52d5ux8207ux95dcux9589}{%
\section{啟動與關閉}\label{ux555fux52d5ux8207ux95dcux9589}}

Windows 的內容

有一張圖片:

\begin{figure}
\centering
\includegraphics{./tex2pdf.33416/e0ef408d9559203849a0aa26f79f9b032b709c7a.png}
\caption{Kmol\label{fig:駱駝}}
\end{figure}

稱為圖 \ref{fig:駱駝}。

各 md 檔案可以在 images 目錄下自訂與 md 檔案名稱相同的子目錄存放影像檔案

\hypertarget{ux555fux52d5ux8207ux95dcux95892}{%
\section{啟動與關閉2}\label{ux555fux52d5ux8207ux95dcux95892}}

\hypertarget{python-ux7a0bux5f0fux8a9eux6cd5}{%
\chapter{Python 程式語法}\label{python-ux7a0bux5f0fux8a9eux6cd5}}

Python 程式語法

\hypertarget{ux8b8aux6578ux547dux540d}{%
\section{變數命名}\label{ux8b8aux6578ux547dux540d}}

IPv4 的內容

有一張圖片:

\begin{figure}
\centering
\includegraphics{./tex2pdf.33416/e0ef408d9559203849a0aa26f79f9b032b709c7a.png}
\caption{Kmol\label{fig:駱駝}}
\end{figure}

稱為圖 \ref{fig:駱駝}。

各 md 檔案可以在 images 目錄下自訂與 md 檔案名稱相同的子目錄存放影像檔案

\hypertarget{print-ux51fdux5f0f}{%
\section{print 函式}\label{print-ux51fdux5f0f}}

\hypertarget{ux91cdux8907ux8ff4ux5708}{%
\section{重複迴圈}\label{ux91cdux8907ux8ff4ux5708}}

\begin{figure}
\centering
\includegraphics{./tex2pdf.33416/4be2397ea792a71e2f4da9500ff7c84d700b58a4.png}
\caption{for迴圈\label{fig:for迴圈}}
\end{figure}

稱為圖 \ref{fig:for迴圈}。

\hypertarget{ux5224ux65b7ux5f0f}{%
\section{判斷式}\label{ux5224ux65b7ux5f0f}}

\hypertarget{ux6578ux5217}{%
\section{數列}\label{ux6578ux5217}}

\hypertarget{pyqt5-ux7c21ux4ecb}{%
\chapter{PyQt5 簡介}\label{pyqt5-ux7c21ux4ecb}}

說明 PyQt5 基本架構與程式開發流程

\hypertarget{pyqt5-ux67b6ux69cb}{%
\section{PyQt5 架構}\label{pyqt5-ux67b6ux69cb}}

C 的內容

其中包含一個表格:

\begin{longtable}[]{@{}ccccccccc@{}}
\caption{Python 網際框架比較 \label{tbl:網際框架}}\tabularnewline
\toprule
\begin{minipage}[b]{0.09\columnwidth}\centering
Framework\strut
\end{minipage} & \begin{minipage}[b]{0.07\columnwidth}\centering
Started\strut
\end{minipage} & \begin{minipage}[b]{0.04\columnwidth}\centering
Py2\strut
\end{minipage} & \begin{minipage}[b]{0.04\columnwidth}\centering
Py3\strut
\end{minipage} & \begin{minipage}[b]{0.04\columnwidth}\centering
ORM\strut
\end{minipage} & \begin{minipage}[b]{0.13\columnwidth}\centering
Template Engine\strut
\end{minipage} & \begin{minipage}[b]{0.11\columnwidth}\centering
Auth Moudule\strut
\end{minipage} & \begin{minipage}[b]{0.12\columnwidth}\centering
Database Admin\strut
\end{minipage} & \begin{minipage}[b]{0.11\columnwidth}\centering
Project Scale\strut
\end{minipage}\tabularnewline
\midrule
\endfirsthead
\toprule
\begin{minipage}[b]{0.09\columnwidth}\centering
Framework\strut
\end{minipage} & \begin{minipage}[b]{0.07\columnwidth}\centering
Started\strut
\end{minipage} & \begin{minipage}[b]{0.04\columnwidth}\centering
Py2\strut
\end{minipage} & \begin{minipage}[b]{0.04\columnwidth}\centering
Py3\strut
\end{minipage} & \begin{minipage}[b]{0.04\columnwidth}\centering
ORM\strut
\end{minipage} & \begin{minipage}[b]{0.13\columnwidth}\centering
Template Engine\strut
\end{minipage} & \begin{minipage}[b]{0.11\columnwidth}\centering
Auth Moudule\strut
\end{minipage} & \begin{minipage}[b]{0.12\columnwidth}\centering
Database Admin\strut
\end{minipage} & \begin{minipage}[b]{0.11\columnwidth}\centering
Project Scale\strut
\end{minipage}\tabularnewline
\midrule
\endhead
\begin{minipage}[t]{0.09\columnwidth}\centering
Pyramid\strut
\end{minipage} & \begin{minipage}[t]{0.07\columnwidth}\centering
2005\strut
\end{minipage} & \begin{minipage}[t]{0.04\columnwidth}\centering
V\strut
\end{minipage} & \begin{minipage}[t]{0.04\columnwidth}\centering
V\strut
\end{minipage} & \begin{minipage}[t]{0.04\columnwidth}\centering
\strut
\end{minipage} & \begin{minipage}[t]{0.13\columnwidth}\centering
\strut
\end{minipage} & \begin{minipage}[t]{0.11\columnwidth}\centering
V\strut
\end{minipage} & \begin{minipage}[t]{0.12\columnwidth}\centering
\strut
\end{minipage} & \begin{minipage}[t]{0.11\columnwidth}\centering
large\strut
\end{minipage}\tabularnewline
\begin{minipage}[t]{0.09\columnwidth}\centering
Django\strut
\end{minipage} & \begin{minipage}[t]{0.07\columnwidth}\centering
2006\strut
\end{minipage} & \begin{minipage}[t]{0.04\columnwidth}\centering
V\strut
\end{minipage} & \begin{minipage}[t]{0.04\columnwidth}\centering
V\strut
\end{minipage} & \begin{minipage}[t]{0.04\columnwidth}\centering
V\strut
\end{minipage} & \begin{minipage}[t]{0.13\columnwidth}\centering
V\strut
\end{minipage} & \begin{minipage}[t]{0.11\columnwidth}\centering
V\strut
\end{minipage} & \begin{minipage}[t]{0.12\columnwidth}\centering
V\strut
\end{minipage} & \begin{minipage}[t]{0.11\columnwidth}\centering
large\strut
\end{minipage}\tabularnewline
\begin{minipage}[t]{0.09\columnwidth}\centering
Flask\strut
\end{minipage} & \begin{minipage}[t]{0.07\columnwidth}\centering
2010\strut
\end{minipage} & \begin{minipage}[t]{0.04\columnwidth}\centering
V\strut
\end{minipage} & \begin{minipage}[t]{0.04\columnwidth}\centering
\strut
\end{minipage} & \begin{minipage}[t]{0.04\columnwidth}\centering
\strut
\end{minipage} & \begin{minipage}[t]{0.13\columnwidth}\centering
\strut
\end{minipage} & \begin{minipage}[t]{0.11\columnwidth}\centering
\strut
\end{minipage} & \begin{minipage}[t]{0.12\columnwidth}\centering
\strut
\end{minipage} & \begin{minipage}[t]{0.11\columnwidth}\centering
small\strut
\end{minipage}\tabularnewline
\bottomrule
\end{longtable}

稱為表 \ref{tbl:網際框架}。

\begin{longtable}[]{@{}lcr@{}}
\caption{價目表 \label{tbl:價目表}}\tabularnewline
\toprule
Tables & Are & Cool\tabularnewline
\midrule
\endfirsthead
\toprule
Tables & Are & Cool\tabularnewline
\midrule
\endhead
col 1 is & left-aligned & \$1600\tabularnewline
col 2 is & centered & \$12\tabularnewline
col 3 is & right-aligned & \$1\tabularnewline
\bottomrule
\end{longtable}

稱為表 \ref{tbl:價目表}。

關於表格生成可以參考這裡:\url{http://www.tablesgenerator.com/markdown_tables}

\hypertarget{calculator-ux7a0bux5f0f}{%
\chapter{Calculator 程式}\label{calculator-ux7a0bux5f0f}}

Calculator 程式細部說明

\hypertarget{ux5efaux7acbux5c0dux8a71ux6846}{%
\section{建立對話框}\label{ux5efaux7acbux5c0dux8a71ux6846}}

自動控制 的內容

其中包含一個表格:

\begin{longtable}[]{@{}ccccccccc@{}}
\caption{Python 網際框架比較 \label{tbl:網際框架}}\tabularnewline
\toprule
\begin{minipage}[b]{0.09\columnwidth}\centering
Framework\strut
\end{minipage} & \begin{minipage}[b]{0.07\columnwidth}\centering
Started\strut
\end{minipage} & \begin{minipage}[b]{0.04\columnwidth}\centering
Py2\strut
\end{minipage} & \begin{minipage}[b]{0.04\columnwidth}\centering
Py3\strut
\end{minipage} & \begin{minipage}[b]{0.04\columnwidth}\centering
ORM\strut
\end{minipage} & \begin{minipage}[b]{0.13\columnwidth}\centering
Template Engine\strut
\end{minipage} & \begin{minipage}[b]{0.11\columnwidth}\centering
Auth Moudule\strut
\end{minipage} & \begin{minipage}[b]{0.12\columnwidth}\centering
Database Admin\strut
\end{minipage} & \begin{minipage}[b]{0.11\columnwidth}\centering
Project Scale\strut
\end{minipage}\tabularnewline
\midrule
\endfirsthead
\toprule
\begin{minipage}[b]{0.09\columnwidth}\centering
Framework\strut
\end{minipage} & \begin{minipage}[b]{0.07\columnwidth}\centering
Started\strut
\end{minipage} & \begin{minipage}[b]{0.04\columnwidth}\centering
Py2\strut
\end{minipage} & \begin{minipage}[b]{0.04\columnwidth}\centering
Py3\strut
\end{minipage} & \begin{minipage}[b]{0.04\columnwidth}\centering
ORM\strut
\end{minipage} & \begin{minipage}[b]{0.13\columnwidth}\centering
Template Engine\strut
\end{minipage} & \begin{minipage}[b]{0.11\columnwidth}\centering
Auth Moudule\strut
\end{minipage} & \begin{minipage}[b]{0.12\columnwidth}\centering
Database Admin\strut
\end{minipage} & \begin{minipage}[b]{0.11\columnwidth}\centering
Project Scale\strut
\end{minipage}\tabularnewline
\midrule
\endhead
\begin{minipage}[t]{0.09\columnwidth}\centering
Pyramid\strut
\end{minipage} & \begin{minipage}[t]{0.07\columnwidth}\centering
2005\strut
\end{minipage} & \begin{minipage}[t]{0.04\columnwidth}\centering
V\strut
\end{minipage} & \begin{minipage}[t]{0.04\columnwidth}\centering
V\strut
\end{minipage} & \begin{minipage}[t]{0.04\columnwidth}\centering
\strut
\end{minipage} & \begin{minipage}[t]{0.13\columnwidth}\centering
\strut
\end{minipage} & \begin{minipage}[t]{0.11\columnwidth}\centering
V\strut
\end{minipage} & \begin{minipage}[t]{0.12\columnwidth}\centering
\strut
\end{minipage} & \begin{minipage}[t]{0.11\columnwidth}\centering
large\strut
\end{minipage}\tabularnewline
\begin{minipage}[t]{0.09\columnwidth}\centering
Django\strut
\end{minipage} & \begin{minipage}[t]{0.07\columnwidth}\centering
2006\strut
\end{minipage} & \begin{minipage}[t]{0.04\columnwidth}\centering
V\strut
\end{minipage} & \begin{minipage}[t]{0.04\columnwidth}\centering
V\strut
\end{minipage} & \begin{minipage}[t]{0.04\columnwidth}\centering
V\strut
\end{minipage} & \begin{minipage}[t]{0.13\columnwidth}\centering
V\strut
\end{minipage} & \begin{minipage}[t]{0.11\columnwidth}\centering
V\strut
\end{minipage} & \begin{minipage}[t]{0.12\columnwidth}\centering
V\strut
\end{minipage} & \begin{minipage}[t]{0.11\columnwidth}\centering
large\strut
\end{minipage}\tabularnewline
\begin{minipage}[t]{0.09\columnwidth}\centering
Flask\strut
\end{minipage} & \begin{minipage}[t]{0.07\columnwidth}\centering
2010\strut
\end{minipage} & \begin{minipage}[t]{0.04\columnwidth}\centering
V\strut
\end{minipage} & \begin{minipage}[t]{0.04\columnwidth}\centering
\strut
\end{minipage} & \begin{minipage}[t]{0.04\columnwidth}\centering
\strut
\end{minipage} & \begin{minipage}[t]{0.13\columnwidth}\centering
\strut
\end{minipage} & \begin{minipage}[t]{0.11\columnwidth}\centering
\strut
\end{minipage} & \begin{minipage}[t]{0.12\columnwidth}\centering
\strut
\end{minipage} & \begin{minipage}[t]{0.11\columnwidth}\centering
small\strut
\end{minipage}\tabularnewline
\bottomrule
\end{longtable}

稱為表 \ref{tbl:網際框架}。

\begin{longtable}[]{@{}lcr@{}}
\caption{價目表 \label{tbl:價目表}}\tabularnewline
\toprule
Tables & Are & Cool\tabularnewline
\midrule
\endfirsthead
\toprule
Tables & Are & Cool\tabularnewline
\midrule
\endhead
col 1 is & left-aligned & \$1600\tabularnewline
col 2 is & centered & \$12\tabularnewline
col 3 is & right-aligned & \$1\tabularnewline
\bottomrule
\end{longtable}

稱為表 \ref{tbl:價目表}。

關於表格生成可以參考這裡:\url{http://www.tablesgenerator.com/markdown_tables}

{建立對話框1\label{fig:建立對話框1}}

稱為圖 \ref{fig:建立對話框1}。

{[}建立建立對話框1{]}{[}{]}

稱為圖 \{\textbf{???}\}。

\begin{figure}
\centering
\includegraphics{./tex2pdf.33416/2a5d343f87e6948e4f400587a8eca053e1d7fa3a.png}
\caption{建立對話框3\label{fig:建立對話框3}}
\end{figure}

稱為圖 \ref{fig:建立對話框3}。

\hypertarget{ux5efaux7acbux6309ux9215}{%
\section{建立按鈕}\label{ux5efaux7acbux6309ux9215}}

\hypertarget{ux5efaux7acbux7a0bux5f0fux78bc}{%
\section{建立程式碼}\label{ux5efaux7acbux7a0bux5f0fux78bc}}

\hypertarget{ux5fc3ux5f97}{%
\chapter{心得}\label{ux5fc3ux5f97}}

期末報告心得

Fossil SCM
Fossil是一個分散式版本控制系統、缺陷跟蹤管理系統以及在軟體開發中使用的wiki軟體伺服器。
對於設計師而言,日常工作中最常使用的工具,可能會是編輯器,或專為某種程式語言所設計的整合開發環境;而對負責軟體開發工作的軟體團隊成員來說,版本控制系統則是另一套相當重要的軟體工具。如果沒有版本控制系統,大型軟體的開發團隊成員將難以有效控制軟體版本,並可能導致程式臭蟲增加。一般人對於軟體本身的使用需求,多半是希望操作越簡單越好,並有相當程度的穩定性與可靠性。而操作簡單與系統本身穩定性高,正是
Fossil
所強調的二大重點。一般人即使沒有使用版本控制軟體的經驗,也能在閱讀
Fossil 提供的簡單文件之後立即上手。Fossil
之所以可以作為官方網站的平台,是因為除了版本控制系統相關的功能以外,亦提供了程式臭蟲追蹤
(Bug Tracking)
與維基共筆系統的支援能力。與本身的版本控制系統功能類似,Fossil
的程式臭蟲追蹤功能與維基共筆系統採用分散式的處理方式。利用 Fossil 作為
Blog
平台的架設解決方案,所以無論使用者需要的是單純的版本控制,或是希望架設網站作為資訊分享的平台,都能利用
Fossil 一併解決。 網誌心得 --- 40623201 -
https://cpb.kmol.info/40623201/doc/trunk/blog/

40623202 - https://cpb.kmol.info/40623202/doc/trunk/blog/

40623203 - https://cpb.kmol.info/40623203/doc/trunk/blog/

40623210 - https://cpb.kmol.info/40623210/doc/trunk/blog/

40623211 - https://cpb.kmol.info/40623211/doc/trunk/blog/

40623212 - https://cpb.kmol.info/40623212/doc/trunk/blog/

\hypertarget{github-ux5354ux540cux5009ux5132}{%
\section{Github 協同倉儲}\label{github-ux5354ux540cux5009ux5132}}

bg1 協同倉儲 : https://github.com/40623203/bg1\_pyqt5\_calculator

\hypertarget{ux5b78ux54e1ux5fc3ux5f97}{%
\section{學員心得}\label{ux5b78ux54e1ux5fc3ux5f97}}

40623201 -

40623202 -

40623203 -

40623210 -

40623211 - 經過這幾週的計算機製作, 我學到如何應用 PyQt 和 eric6,
更學到分工合作的重要, 有效的分工, 能讓工作流程更順暢, 各個組員盡守本分,
使我們更團結一心, 其中我在製作根號時, 因爲一個英文字母漏掉,
導致系統無法運算, 所以我看到電腦打字準確性的重要,
第一次自己製作計算機是一個好棒的體驗

40623212 -

說明各學員任務與執行過程

40623201 -

40623202 -

40623203 -

40623210 -

40623211 -

40623212 -

\hypertarget{ux7d50ux8ad6}{%
\chapter{結論}\label{ux7d50ux8ad6}}

期末報告結論

\hypertarget{ux7d50ux8ad6ux8207ux5efaux8b70}{%
\section{結論與建議}\label{ux7d50ux8ad6ux8207ux5efaux8b70}}

結論與建議內容

\hypertarget{ux53c3ux8003ux6587ux737b}{%
\chapter{參考文獻}\label{ux53c3ux8003ux6587ux737b}}


\end{document}
